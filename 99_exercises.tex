\documentclass[aspectratio=169,unicode,dvipdfmx,14pt]{beamer}


\usepackage{url}
\usepackage{bm}
\usepackage{amsmath}
\usepackage{amssymb}
\usepackage{mathtools}
\usepackage{graphicx}
\usepackage[absolute,overlay]{textpos}
\usepackage{hyperref}
\usepackage{listings}
\usepackage{changepage}
\usepackage{lipsum}
\usepackage{animate}
\usepackage{bbm}

\usefonttheme[onlymath]{serif}

\DeclareMathOperator*{\argmax}{argmax}

\DeclarePairedDelimiterX{\infdivx}[2]{(}{)}{%
  #1\;\delimsize\|\;#2%
}
\newcommand{\infdiv}{D_{\scriptsize \mbox{KL}}\infdivx}
\DeclarePairedDelimiter{\norm}{\lVert}{\rVert}

\hypersetup{
	setpagesize=false,
	bookmarksnumbered=true,%
	bookmarksopen=true,%
	colorlinks=true,%
	linkcolor=blue,
	citecolor=red,
}

\newcommand\FontMath{\fontsize{10}{12}\selectfont}
\renewcommand{\baselinestretch}{1.3}
\renewcommand{\familydefault}{\sfdefault}
\renewcommand{\kanjifamilydefault}{\gtdefault}
\usepackage[deluxe, expert]{otf}

\setbeamertemplate{navigation symbols}{}
\setbeamertemplate{footline}[frame number]
\setbeamerfont{footline}{size={\fontsize{15}{15}}}

\setbeamerfont{author}{size=\Large}
\setbeamerfont{institute}{size=\normalsize\itshape}
\setbeamerfont{title}{size=\huge}
\setbeamerfont{subtitle}{size=\LARGE\normalfont\slshape}


\title{ \\5つ目の課題の答え}
\author{\texorpdfstring{正田 備也\newline\href{mailto:masada@rikkyo.ac.jp}{masada@rikkyo.ac.jp}}{正田 備也}}
\date{}

\begin{document}

\begin{frame}
\titlepage
\end{frame}

\begin{frame}{ガンマ関数の性質}
授業中にも述べましたが、ガンマ関数$\Gamma(x)$について、次の等式が成立します。
\begin{align}
\Gamma(x+1) = x \Gamma(x)
\label{eq1}
\end{align}
このことを、以下の議論で使います。
\end{frame}


\begin{frame}{ベータ分布の規格化定数}
\FontMath
また、ベータ分布の密度関数$f(\mu;\alpha,\beta)=\frac{\Gamma(\alpha+\beta)}{\Gamma(\alpha)\Gamma(\beta)}
\mu^{\alpha-1}(1 - \mu)^{\beta - 1}$を$[0,1]$の範囲で積分すると$1$になること、つまり
\begin{align}
1 = \int_0^1 \frac{\Gamma(\alpha+\beta)}{\Gamma(\alpha)\Gamma(\beta)}
\mu^{\alpha-1}(1 - \mu)^{\beta - 1} d\mu
\label{eq2}
\end{align}
であることも、以下の議論では利用します。

(これは、積分すると$1$になるようにするには、$\mu^{\alpha-1}(1 - \mu)^{\beta - 1}$に$\frac{\Gamma(\alpha+\beta)}{\Gamma(\alpha)\Gamma(\beta)}$をかけ算しておく必要がある、ということでもあります。)

特に、上の式\eqref{eq2}を変形すると
\begin{align}
\frac{\Gamma(\alpha)\Gamma(\beta)}{\Gamma(\alpha+\beta)} = \int_0^1 
\mu^{\alpha-1}(1 - \mu)^{\beta - 1} d\mu
\label{eq3}
\end{align}
となることに注意しましょう。
\end{frame}


\begin{frame}{問1}
\FontMath
\vspace{-.2in}
\begin{align}
\int_0^1 \mu f(\mu;\alpha,\beta) d\mu 
& = \int_0^1 \mu \frac{\Gamma(\alpha+\beta)}{\Gamma(\alpha)\Gamma(\beta)}
\mu^{\alpha-1}(1 - \mu)^{\beta - 1} d\mu
\notag \\
& = \int_0^1 \frac{\Gamma(\alpha+\beta)}{\Gamma(\alpha)\Gamma(\beta)}
\mu^{\alpha}(1 - \mu)^{\beta - 1} d\mu
&& \mbox{$\mu$と$\mu^{\alpha-1}$とをまとめた}
\notag \\
& = \frac{\Gamma(\alpha+\beta)}{\Gamma(\alpha)\Gamma(\beta)} \int_0^1 
\mu^{\alpha}(1 - \mu)^{\beta - 1} d\mu
&& \mbox{定数部分を積分の外に出した}
\notag \\
& = \frac{\Gamma(\alpha+\beta)}{\Gamma(\alpha)\Gamma(\beta)} 
\frac{\Gamma(\alpha+1)\Gamma(\beta)}{\Gamma(\alpha+\beta+1)}
&& \mbox{式\eqref{eq3}の$\alpha$を$\alpha+1$で置き換えて使った}
\notag \\ 
& = \frac{\Gamma(\alpha+1)}{\Gamma(\alpha)}
\frac{\Gamma(\alpha+\beta)}{\Gamma(\alpha+\beta+1)}
&& \mbox{$\Gamma(\beta)$を分母分子でキャンセル}
\notag \\ 
& = \frac{\alpha\Gamma(\alpha)}{\Gamma(\alpha)}
\frac{\Gamma(\alpha+\beta)}{(\alpha+\beta)\Gamma(\alpha+\beta)}
&& \mbox{式\eqref{eq1}より}
\notag \\
& = \frac{\alpha}{\alpha+\beta}
\end{align}
\end{frame}

\begin{frame}{問2}
\FontMath
(事後分布)∝(尤度)×(事前分布)なので、規格化定数を除き、$\mu$に依存する部分だけに着目すると
\begin{align}
\mbox{(事後分布)} & \propto \mu^m (1- \mu)^l \times \mu^{\alpha-1}(1 - \mu)^{\beta - 1}
\notag \\
& = \mu^{\alpha + m -1}(1 - \mu)^{\beta + l - 1}
\end{align}
これは、パラメータが$\alpha + m$と$\beta + l$のベータ分布$\mbox{Beta}(\alpha + m,\beta + l)$である。

よってその密度関数は、規格化定数もきちんと書くと
\begin{align}
\frac{\Gamma(\alpha + m + \beta + l)}{\Gamma(\alpha+m)\Gamma(\beta+l)}
\mu^{\alpha + m -1}(1 - \mu)^{\beta + l - 1}
\end{align}
となる。
\end{frame}


\end{document}